\documentclass[12pt]{report}
\usepackage[utf8]{inputenc}
\usepackage{listings}
\usepackage{pdfpages}
\usepackage{graphicx}
\graphicspath{ {images/} }
\usepackage{svg}
\usepackage{amsmath}
\usepackage{amssymb}
\usepackage{booktabs}
\usepackage{extarrows}
\usepackage[ruled,linesnumbered]{algorithm2e}

\lstdefinestyle{refactor} {
    keywordstyle=\ttfamily\color{green}\ttfamily,
    stringstyle=\sffamily\color{black},
    morekeywords={Seq,Pipe,Farm,Comp,Map},
}

\begin{document}	
\title{}

\maketitle

\section{Grammar and Parser}
The grammar defines basic constructs of patterns and it is used as an input to parser generator ANTLR(ANother Tool for Language Recognition). ANTLR generates a parser which translates input into a tree of skeletons. example input:

\begin{lstlisting}[style=refactor, caption={example input program},label={code1}]
a = Seq (10);
b = Seq (20);
main = Pipe(Farm(a),b);
\end{lstlisting}
The parser creates a tree with root node Pipe skeleton having two stages Farm and Sequential; Farm stage will have one worker process which is type of Sequential. the resulting tree is passed to the refactoring process which is based on  breadth first search algorithm. the input tree is expanded into a forest of trees in the refactoring process. The tree expansion is done by implementing rewriting rules at level of the skeletons; a visitor based rewriter visits each skeleton node of a tree  and creates a new tree by applying rewriting rules. 

here are the basic rewriting rules:

\begin{itemize}
\item $\Delta \xLongleftrightarrow[\text{Farm\_Intro}]{\text{Farm\_Elim}} farm (\Delta)$\\
\item $Comp(\Delta_1, \Delta_2 )\xLongleftrightarrow[\text{Pipe\_Intro}]{\text{Pipe\_Elim}} Pipe (\Delta_1,\Delta_2)$\\
\item $Map(\Delta) \xLongrightarrow[\text{Map\_Elim}]{\text{}}  (\Delta)$\\
\item $Comp(Comp(\Delta_1, \Delta_2 ),\Delta_3)\xLongleftrightarrow[\text{Comp\_Assoc}]{\text{}} Comp (\Delta_1,Comp(\Delta_2,\Delta_3))$\\
\item $Pipe(Pipe(\Delta_1, \Delta_2 ),\Delta_3)\xLongleftrightarrow[\text{Pipe\_Assoc}]{\text{}} Pipe (\Delta_1,Pipe(\Delta_2,\Delta_3))$\\
\item $Map(Pipe(\Delta_1, \Delta_2 ))\xLongleftrightarrow[\text{Pipe\_of\_map}]{\text{Map\_of\_Pipe}} Pipe (Map(\Delta_1),Map(\Delta_2))$\\
\item $Map(Comp(\Delta_1, \Delta_2 ))\xLongleftrightarrow[\text{Comp\_of\_map}]{\text{Map\_of\_Pipe}} Comp (Map(\Delta_1),Map(\Delta_2))$\\
\end{itemize}


\section{refactoring algorithm}
it starts by visiting the root node and proceeds to the child nodes . each visiting operation generates a set of trees ; they indicate different rewriting options for a particular pattern tree.each newly created tree is inserted into queue so that it'll be processed by the visitor. the algorithm is based on breadth-first search at the tree level and visitor pattern at the node level. 

\begin{lstlisting}[caption={skeleton tree}, label={tree1}]
	Skeleton {
		Skeleton root;
		List<Skeleton> children;
		List<Skeleton> reWritngOptions;
		ReWringRule rule;
	}
	Edge{
		Skeleton from;
		Skeleton to;
		ReWritingRule rule;
	}
\end{lstlisting}
\SetKwRepeat{Do}{do}{while}
\begin{algorithm}[H]
\SetAlgoLined
 \caption {refactoring algorithm}\label{euclid}
  input: program to be parallelized\\
  \KwResult{directed graph of the rewriting options }
  building skeleton tree \texttt{st}\;
  initialization\;
  add \texttt{st to queue}\\
  \While{queue is not empty}{
     st =queue.remove()\;
    patterns = refactor(st)\;
    \Repeat{all patterns are inserted}{
	     If{ pattern not in queue \& pattern.height  $\leq$ maxHeight}{
	        queue.add(pattern)\;  
		}
	}
	\Repeat{all stages are refactored}{
		 patterns = refactor (stage)\;
	     If{ pattern not in queue \& pattern.height $\leq$ maxHeight}{
	        queue.add(pattern)\;
	     }      
	}  
  }

\end{algorithm}
line 1: input, the algorithm accepts input in format of  Listing \ref{code1}.\\
line 2: building tree; the input code is parsed and a new parse tree is constructed .the parser generated by ANTLR parse and construct a tree from the input code.\\
line3: initialization; initlialize queue for the breadth first search, directed graph to hold the trees generated by the expansion process\\
line 4: start breadth first search\\
line 7-9: refactoring; this is implemented by using visitor pattern which visits every node of the tree staring from the root and each visit operation creates a new tree which in turn will be added into the queue for further refactoring. since this process creates infinite number of trees i have added a condition to stop the process. for a tree to be refactored it should have a height less than maxHeight.
whenever a new alrernative rewriting option is constructed the algoithm will add it to the graph if it is not already present and it creates an edge from the original skeleton tree to the new one labeling it with the rewring rule used to generate it. if the tree already exists it will just create a new edge between the original tree and the the existing tree object with the rule. \\
line 10-14 is similar to above process except it works on the child nodes of the tree.

\section{non functional parameters}
i have focused mainly on the service time and parallelism degree of the skeletons. 
while creating trees of refactoring patterns is straight forward , allocating resources is a very difficult problem to tackle. here is how  paralellism degree and service times are computed for each kind of skeleton:\\
	let \texttt{ Tscatter 	T\textsubscript{s}, 					Tcollector T\textsubscript{c} ,Temitter T\textsubscript{e}, Tgather T\textsubscript{g}=1\\
	$T_{\Delta}$ is service time of the stage or worker skeleton\\
	Sequential Skeleton:	n = 1; Ts = ts\\
	Farm Skeleton:	the ideal parallelism degree is computed by $\frac{T\textsubscript{e}}{T_{\Delta}}$\\
	Pipeline Skeleton: the ideal parallelism degree is the number of stages\\
	Comp Skeleton:\\
     Map Skeleton:  n = $\sqrt{\frac{T_{\Delta}}{max(T_s, T_g)}}$}\\
Farm:\\
	the ideal service time is calculated using the formula: \texttt {Max($T_e,T_c,\frac{T_\Delta}{n})$}\\
Pipeline:\\
	\texttt{$T_s = Max(T_{si}) \forall i \in pipe stages$}


\end{document}